\documentclass[11pt,letterpaper]{article}
\usepackage[spanish]{babel}
\usepackage{amsmath}
\usepackage{amsfonts}
\usepackage{amssymb}
\usepackage{makeidx}
\usepackage{graphicx}
\usepackage[left=2cm,right=2cm,top=2cm,bottom=2cm]{geometry}

\title{Propuesta de proyecto de prácticas: caminatas cuánticas con decoherencia}
\author{José Alfredo de León}

\begin{document}
\maketitle

\begin{abstract}
Se presenta una propuesta de proyecto de prácticas para un estudiante de
física de la ECFM de octavo semestre como mínimo para estudiar el efecto
de la decoherencia en caminatas cuánticas discretas en 1 dimensión.
\end{abstract}

\section{Objetivo principal}
Estudiar cómo se modifican las caminatas cuánticas discretas en 1 dimensión
(``DQWL'' por sus siglas en inglés: \textit{discrete quantum walks in a line})
bajo la acción de decoherencia, modelada con canales cuánticos Pauli 
component erasing.

\section{Objetivos específicos}
\begin{itemize}
\item Estudiar el formalismo de la matriz de densidad. 
\item Estudiar la teoría de los canales cuánticos y los canales cuánticos Pauli component erasing. 
\item Estudiar el modelo de DQWL.
\item Hacer un revisión bibliográfica del estado del arte en modelos de caminatas aleatorias con decoherencia. 
\item Implementar numéricamente un modelo de DQWL con decoherencia modelada 
con Pauli component erasing channels.
\end{itemize}

\section{Duración}
El proyecto se pretende que sea finalizado en 1 semestre (4 meses), considerando
como finalizado el proyecto hasta la entrega del informe final. Se espera 
que la o el estudiante dedique 6 horas como máximo a la semana al proyecto. 

\section{Descripción del proyecto}

El proyecto consiste en tres partes, no todas igual de extensas: 
(1) el estudiante
estudiará el marco teórico, 
que se compone del formalismo de la matriz de densidad,
la teoría de los canales cuánticos y las caminatas cuánticas discretas en 1 
dimensión; (2) el estudiante hará una revisión bibliográfica del estado del
arte de las caminatas cuánticas aleatorias; (3) finalmente, el estudiante
implementará numéricamente un modelo de decoherencia en caminatas cuánticas
discretas en 1 dimensión y estudiará cómo se modifican en contraste con 
caminatas sin decoherencia. Se estipula que cada parte del proyecto consista
en el $50\%,\,25\%$ y $25\%$, respectivamente.

Ahora se describen con un más detalle las partes 1 y 2. En la primera parte,
el estudiante estudiará el formalismo de la matriz de densidad y la teoría 
de los canales cuánticos de los capítulos 2 y 8 del libro introductorio~\cite{nielsen_chuang_2010} de Nielsen y Chuang, además 
estudiará la forma de Kraus de los canales \textit{Pauli component erasing}, que 
modelan decoherencia de sistemas de partículas dos niveles, del artículo~\cite{PCE2022}; por último, estudiará 
las caminatas cuánticas discretas en 1 dimensión del artículo de
revisión~\cite{venegas2012quantum} de Venegas-Andraca. Para la segunda parte, 
se orientará al estudiante para que aprenda a hacer un revisión bibliográfica
y que lo ponga en práctica investigando el estado del conocimiento 
de los modelos de decoherencia
en caminatas cuánticas. El acceso a los artículos de revistas se hará por 
medio de la cuenta institucional de la UNAM del asesor. El estudiante 
escribirá en el informe final sobre lo aprendido del marco teórico y 
sobre su revisión bibliográfica.

Finalmente, se describe en breve la última parte del proyecto. El objetivo 
de esta parte del proyecto es que el estudiante ponga inmediatamente en
práctica lo aprendido en las primeras dos partes. El estudiante 
creará sus propias herramientas computacionales, en el lenguaje de programación
de su preferencia, para modelar caminatas aleatorias discretas en 1 dimensión 
con decoherencia. Él o ella estudiará sistemáticamente cómo se modifican 
las caminatas cuánticas discretas en 1 dimensión bajo la acción de los 
canales Pauli component erasing. Sus resultados y una discusión deberán 
ser incluídos en la última parte del informe final.

\section{Sobre la dinámica de asesoría}
El asesor se compromete a orientar durante todo el proyecto al estudiante
y a reunirse de manera virtual con él o ella por lo menos 1 vez cada 2
semanas. Se pretende que en estas reuniones periódicas se resuelvan 
dudas del estudiante y que él o ella presente sus avances de estudio, revisión bibliografía o implementación numérica,
para así asegurar la finalización del proyecto en el tiempo estipulado.


\bibliographystyle{amsplain}
\bibliography{references}

\end{document}