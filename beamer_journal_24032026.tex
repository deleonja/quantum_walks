\documentclass{beamer}
\usetheme{metropolis} % O el que prefieras
\usepackage{amsmath, amssymb}

\title{Una caminata cuántica es un autómata celular cuántico (QCA)}
\author{}
\date{}

\begin{document}

\maketitle

\begin{frame}{¿Qué define a un QCA? (Axiomas de Schumacher-Werner)}
Un autómata celular cuántico (QCA), satisface:
  
\begin{enumerate}
  \item \textbf{Localidad:} El estado de una celda en $t+1$ solo depende de su vecindad en $t$.
  
  \item \textbf{Unitariedad.} La evolución está descrita por $U$ ($UU^\dagger=I$).
  
  \item \textbf{Homogeneidad:} Las reglas son las mismas en todo el espacio, i.e., $U$ tiene invarianza translacional.
  
  \item \textbf{Discretización:} El sistema vive en una red (espacio discreto) y evoluciona en pasos de tiempo definidos $\Delta t$.
\end{enumerate}
\end{frame}

\begin{frame}{Quantum Game of Life}
  Un ejemplo de QCA...
\end{frame}

\begin{frame}{Caminata cuántica con 1 partícula}
  XXZ en el subespacio de 1 partícula 
\end{frame}

\begin{frame}{}
  El modelo XXZ original es continuo, pero un QCA es \textbf{discreto}.
  
  \begin{itemize}
    \item \textbf{Hamiltoniano:} $H = \sum_{i} [ J(\sigma_i^x \sigma_{i+1}^x + \sigma_i^y \sigma_{i+1}^y) + \Delta \sigma_i^z \sigma_{i+1}^z ]$
    \item \textbf{Discretización:} Dividimos el tiempo en pasos $\delta t$.
    \item \textbf{Aproximación de Trotter:} 
    $$U(\delta t) \approx e^{-i H_{odd} \delta t} e^{-i H_{even} \delta t} = U_{odd} U_{even}$$
  \end{itemize}
  
  \textbf{Capa Par ($U_{even}$):} Interacciones en enlaces (1,2), (3,4)...
  
  \textbf{Capa Impar ($U_{odd}$):} Interacciones en enlaces (2,3), (4,1)...
\end{frame}  

\begin{frame}{4. La Matriz del "Ladrillo" $u$}
  El operador de dos sitios $u = e^{-i H_{i,i+1} \delta t}$ en la base $\{|00\rangle, |01\rangle, |10\rangle, |11\rangle\}$:
  $$u = \begin{pmatrix} 
    e^{-i\Delta \delta t} & 0 & 0 & 0 \\ 
    0 & \text{cos}\theta e^{i\phi} & -i\text{sin}\theta e^{i\phi} & 0 \\ 
    0 & -i\text{sin}\theta e^{i\phi} & \text{cos}\theta e^{i\phi} & 0 \\ 
    0 & 0 & 0 & e^{-i\Delta \delta t} 
  \end{pmatrix}$$
  \begin{itemize}
    \item \textbf{Ceros:} Garantizan que el autómata conserva magnetización.
    \item \textbf{Bloque Central:} Define la amplitud de probabilidad del "paso" en la caminata.
  \end{itemize}
\end{frame}
\end{document}